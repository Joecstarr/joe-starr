\documentclass[14pt]{extarticle}
\usepackage[margin=.75in]{geometry}

\usepackage{svg}
\usepackage{amsmath, amssymb, amsthm, amsfonts, mathrsfs}
\usepackage{times, flexisym, mdframed, xcolor}
\usepackage{ulem,multicol}
\usepackage{mathtools}
\usepackage{tikz}
\usepackage{hyperref}
\usepackage{graphicx}
\usepackage{fancyhdr}
\usepackage{tikz-cd}
\usepackage[doublespacing]{setspace}
\usepackage[shortlabels]{enumitem}

\newcommand{\nd}{\noindent}
\newcommand{\Vspc}{\vspace*{0.1in}}
\newcommand{\vhalfpg}{\vspace{5in}}
\newcommand{\vthirdpg}{\vspace{3in}}
\newcommand{\vquartpg}{\vspace{2in}}

% --------------------------------------------------------------
%                         Config Options
% --------------------------------------------------------------

% \def\lined{a}

% --------------------------------------------------------------
%                         Style options
% --------------------------------------------------------------

\usetikzlibrary{decorations.markings,calc}




% --------------------------------------------------------------
%                         Paired delimiters
% --------------------------------------------------------------
\newcommand{\LP}{\left(}
\newcommand{\RP}{\right)}
\newcommand{\LS}{\left\lbrace}
\newcommand{\RS}{\right\rbrace}
\newcommand{\LB}{\left[}
\newcommand{\RB}{\right]}
\newcommand{\LA}{\left\langle}
\newcommand{\RA}{\right\rangle}
\newcommand{\MM}{\ \middle|\ }
\newcommand{\lrb}[1]{\left[#1\right]}
\newcommand{\lrp}[1]{\left(#1\right)}
\newcommand{\lrs}[1]{\left\{#1\right\}}
\newcommand{\lra}[1]{\left\langle#1\right\rangle}

% --------------------------------------------------------------
%                         New Commands
% --------------------------------------------------------------
\newcommand{\N}{\mathbb{N}}
\newcommand{\Z}{\mathbb{Z}}
\newcommand{\Q}{\mathbb{Q}}
\newcommand{\R}{\mathbb{R}}
\newcommand{\m}{\scalebox{0.5}[1.0]{$-$}}
\newcommand{\img}[1]{\begin{aligned}
    &\ \\
    &\includesvg[width=.5cm]{#1}\\
    &\
\end{aligned}}
\newcommand{\bkt}[1]{\LA\img{#1}\RA}


% --------------------------------------------------------------
%                         Math Operators
% --------------------------------------------------------------
\DeclareMathOperator{\Aut}{Aut}
\DeclareMathOperator{\Syl}{Syl}
\DeclareMathOperator{\Inn}{Inn}


% --------------------------------------------------------------
%                         Renew Commands
% --------------------------------------------------------------
% \renewcommand{\headrulewidth}{0pt}


\begin{document}

\pagestyle{fancy}
\fancyhf{}
\lhead{Starr}

\rhead{Constructing the Jones polynomial to save the world}

\begin{enumerate}
\item{
Using what we discoverd about the type II move deduce that:
\vspace{-2cm}
\begin{center}
\scalebox{2}[2]{
$\bkt{1.svg}=\bkt{6.svg}$
}
\end{center}
}
\vspace{5cm}

\item{
Compute bracket for the other type one move:
\vspace{-2cm}
\begin{center}
\scalebox{2}[2]{
    ${\LA\img{t1.svg}\RA=?\LA\img{t1_2.svg}\RA}$
    }
\end{center}
}
\newpage


\item{
Compute the writhe of:
\begin{center}
    \scalebox{1}[1]{
        $\includesvg[width=4cm]{trefoil.svg}$
        }
\end{center}}

\vspace{5cm}
\item{
Verify that our rule works for the other type I move:
\vspace{-1cm}
\begin{center}
    \scalebox{2}[2]{
        $-A^{-3w\LP \includesvg[width=.3cm]{t1.svg}\RP}\LA\img{t1.svg}\RA=\LA\img{t1_2.svg}\RA$
        }
\end{center}}
\newpage
\item{
Compute the bracket for our anti-knot:
\vspace{-3cm}
\begin{center}
    \scalebox{3}[3]{
        $-A^{-3w\LP \includesvg[width=.3cm]{right.svg}\RP}\LA\img{right.svg}\RA$
        }
\end{center}}
\end{enumerate}
\newpage
\section*{Reference:}
\begin{enumerate}
        \item{ \scalebox{1.5}[1.5]{${\LA \img{unknot.svg} \RA=1}$}\vspace{-2cm}}
        \item{ \scalebox{1.5}[1.5]{$\LA \img{crossing_un.svg}\RA=A\LA \img{6a.svg} \RA+A^{-1}\LA\img{6b.svg}\RA$}\vspace{-2cm}}
        \item{ \scalebox{1.5}[1.5]{${\LA D \sqcup \img{unknot.svg} \RA=\LP-A^{-2}-A^2\RP\LA D\RA}$}\vspace{-1cm}}
        \item{ \scalebox{1.5}[1.5]{$-A^{-3w\LP D\RP}\LA D\RA$}}
\end{enumerate}
\end{document}
